\documentclass{article}
\title{The Oumupo, between music and math}
\author{M Andreatta, M Granger, T Johnson and V Villenave
(tr M Solomon)}
\begin{document}
\maketitle
\begin{abstract}
What if music was, at the end of the day, an accessible, fun and expressive way to engage with math? The musicians of the OuMuPo (Ouvroir de Musique Potentielle, or The Workshop for Potential Music) lead us through this open question via different exercises and experiments.
\end{abstract}

In 1960, François le Lionnais and Raymond Queneau founded a collective whose singular objective was to reinvigorate literary forms. This collective, called the OuLiPo (Ouvroir de Littérature Potentielle, or The Workshop for Potential Literature), continues to existe today and is the source of audacious and provocative work such as Queneau’s Hundred Thousand Billion Poems and Perec’s A Void. Since the group’s inception, François Le Lionnais imagined extending the scope of this group to various other disciplines and, in this spirit, founded several OuXPo’s (Ouvroir de X Potentiel(le), or The Workshop for Potential
X).


The \textbf{Ouvroir de Musique Potentielle}, an active and constantly evolving group, is resolutely part of this tradition.

The OuXPo is concerned with inventing new structures and forms, but at a smaller scale, their experiments bring them sometimes to the reinvention of the language through which their art is expressed.  For example, a writer that forbids the use of certain letters or groups of letters is forced to express herself with a diminished lexicon, often needing to resort to unknown words and archaic formulations that surprise the reader.  The words take on a quality as signifiers (meaning that they contain value in and of themselves as objects of beauty) in addition to, or even before, inducing thought about that which they signify.

In this sense, the OuMuPo is not only brought to reevaluate purely musical constructs (harmony, rhythm, melody, timbre) but also to draw parallels between these devices and other areas of study : texts, graphic arts, geometry and numbers.  The music is, as many know, and excellent and accessible way to wrap one’s head around abstract notions and mathematical problems, some of which are quite challenging.  In a reciprocal manner, music is a constant source of inspiration in the elaboration of new musical processes.  Here are several examples:

\section*{Coding and translation}
Music is in and of itself a mathematical language. Numbers are omnipresent in the musical field and are expressed in various forms:

\begin{itemize}
\item \textbf{rhythms} including the duration of notes and rests, but as
well the number of repetitions of a note or a group of notes.  This even
extends to the quantity of events within a single structural unit (for
example, the number of notes in a measure).
\item \textbf{absolute pitch}, or the degrees of the diatonic scale that are
often numbered with Roman Numerals or designated with letters of the
alphabet. Since the twentieth century, it is even possible to designate a
pitch by its periodic frequency in Hz!
\item \textbf{relative pitch}, or the interval between a pitch and another
one, played either disjointly or simultaneously in time. By counting the
number of semitones between two notes, it is possible to establish detailed
metrics of chords and harmonies. The \emph{Tonnetz} (see below) is, in this
line of thought, an original and interesting tool.
\end{itemize}
In many cases, the transation of a number or a mathematical operation
into musical elements.  This leads to the questioning of the \textbf{numeric
base} being used.  While we are used to manipulating numbers in a base 10
scheme, musical organization can be conceived in entirely different ways.
\begin{itemize}
\item Beats and measures are often counted in units that are $2^x$, leading
to a measurement system in base 4 (the bread-and-butter of European Common Practice
musical phrasing) or 8 (the ``count to eight'' common for dancers).
\item The diatonic scale, an omnipresent seven-note construct in many different musical
traditions including the Western one. The pentatonic scale, which comprises
five notes, is found in many Eastern cultures.
\item The chromatic scale, used since the time of Pythagorian theorists,
requires the partitioning of a musical octave in a twelve-semitone scale.
Today, this scale is a strict geometric series (each note incrementing by a
factor of $2^{1/12}$).
\end{itemize}
It is nevertheless possible to cheat through various means of musical
cunning.  For example, a ten-note set can be derived from the twelve-note
integrality by omitting two notes from the chromatic collection.
\section*{Melodic transformation}
Forms such as the canon have existed for several centuries.  Numerous
composers, including the well-known J.S. Bach, have proven that it is
possible to make musical counterpoint by playing the same melody
simultaneously at different tempos.  With this in mind, we attempt to
explore new melodic and polyphonic structures.  One example of this is
self-similar melodies, meaning melodies that reproduce themselves at
different scales.  In mathematics, this idea is encapsulated by fractals. 
Concretely, the melody rearticulates itself by only playing one note out of
every \emph{n} notes.  In the most complex cases, \emph{n} can change but
the melody remains invariant.
\section*{Combinatorial procedures}
Music, compared to literature, is combinatorially agnostic, meaning that
the constraint of meaning attributed to words that make certain literary
combinations valid and others not is more difficult to articulate in music. 
For example, the French word \emph{mode} can have its letters reorganized
into \emph{d\^{o}me} or \emph{d\'{e}mo} whereas \emph{doem} or \emph{medo}
are not words in the French lexicon.  Musical notes, on the other hand, can
be combined in any order and carry a subjective argument in musical time.
Several classic mathematic tools can be used to articulate musics
combinatorial potential: magic squares, Pascal's triangle, the Dragon's
Curve (see below) and many others.
\end{document}