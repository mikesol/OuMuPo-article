\documentclass{article}
\usepackage{listings}
\newcommand{\Tonnetz}{\emph{Tonnetz}}
\newcommand{\Hamiltonian}{Hamiltonian}
\title{The Oumupo, between music and math}
\author{M Andreatta, M Granger, T Johnson and V Villenave
(tr M Solomon)}
\begin{document}
\maketitle
\begin{abstract}
What if music was, at the end of the day, an accessible, fun and expressive way to engage with math? The musicians of the OuMuPo (Ouvroir de Musique Potentielle, or The Workshop for Potential Music) lead us through this open question via different exercises and experiments.
\end{abstract}

In 1960, François le Lionnais and Raymond Queneau founded a collective whose singular objective was to reinvigorate literary forms. This collective, called the OuLiPo (Ouvroir de Littérature Potentielle, or The Workshop for Potential Literature), continues to existe today and is the source of audacious and provocative work such as Queneau’s Hundred Thousand Billion Poems and Perec’s A Void. Since the group’s inception, François Le Lionnais imagined extending the scope of this group to various other disciplines and, in this spirit, founded several OuXPo’s (Ouvroir de X Potentiel(le), or The Workshop for Potential
X).


The \textbf{Ouvroir de Musique Potentielle}, an active and constantly evolving group, is resolutely part of this tradition.

The OuXPo is concerned with inventing new structures and forms, but at a smaller scale, their experiments bring them sometimes to the reinvention of the language through which their art is expressed.  For example, a writer that forbids the use of certain letters or groups of letters is forced to express herself with a diminished lexicon, often needing to resort to unknown words and archaic formulations that surprise the reader.  The words take on a quality as signifiers (meaning that they contain value in and of themselves as objects of beauty) in addition to, or even before, inducing thought about that which they signify.

In this sense, the OuMuPo is not only brought to reevaluate purely musical constructs (harmony, rhythm, melody, timbre) but also to draw parallels between these devices and other areas of study : texts, graphic arts, geometry and numbers.  The music is, as many know, and excellent and accessible way to wrap one’s head around abstract notions and mathematical problems, some of which are quite challenging.  In a reciprocal manner, music is a constant source of inspiration in the elaboration of new musical processes.  Here are several examples:

\section*{Coding and translation}
Music is in and of itself a mathematical language. Numbers are omnipresent in the musical field and are expressed in various forms:

\begin{itemize}
\item \textbf{rhythms} including the duration of notes and rests, but as
well the number of repetitions of a note or a group of notes.  This even
extends to the quantity of events within a single structural unit (for
example, the number of notes in a measure).
\item \textbf{absolute pitch}, or the degrees of the diatonic scale that are
often numbered with Roman Numerals or designated with letters of the
alphabet. Since the twentieth century, it is even possible to designate a
pitch by its periodic frequency in Hz!
\item \textbf{relative pitch}, or the interval between a pitch and another
one, played either disjointly or simultaneously in time. By counting the
number of semitones between two notes, it is possible to establish detailed
metrics of chords and harmonies. The \Tonnetz (see below) is, in this
line of thought, an original and interesting tool.
\end{itemize}
In many cases, the transation of a number or a mathematical operation
into musical elements.  This leads to the questioning of the \textbf{numeric
base} being used.  While we are used to manipulating numbers in a base 10
scheme, musical organization can be conceived in entirely different ways.
\begin{itemize}
\item Beats and measures are often counted in units that are $2^x$, leading
to a measurement system in base 4 (the bread-and-butter of European Common Practice
musical phrasing) or 8 (the ``count to eight'' common for dancers).
\item The diatonic scale, an omnipresent seven-note construct in many different musical
traditions including the Western one. The pentatonic scale, which comprises
five notes, is found in many Eastern cultures.
\item The chromatic scale, used since the time of Pythagorian theorists,
requires the partitioning of a musical octave in a twelve-semitone scale.
Today, this scale is a strict geometric series (each note incrementing by a
factor of $2^{1/12}$).
\end{itemize}
It is nevertheless possible to cheat through various means of musical
cunning.  For example, a ten-note set can be derived from the twelve-note
integrality by omitting two notes from the chromatic collection.
\section*{Melodic transformation}
Forms such as the canon have existed for several centuries.  Numerous
composers, including the well-known J.S. Bach, have proven that it is
possible to make musical counterpoint by playing the same melody
simultaneously at different tempos.  With this in mind, we attempt to
explore new melodic and polyphonic structures.  One example of this is
self-similar melodies, meaning melodies that reproduce themselves at
different scales.  In mathematics, this idea is encapsulated by fractals. 
Concretely, the melody rearticulates itself by only playing one note out of
every \emph{n} notes.  In the most complex cases, \emph{n} can change but
the melody remains invariant.
\section*{Combinatorial procedures}
Music, compared to literature, is combinatorially agnostic, meaning that
the constraint of meaning attributed to words that make certain literary
combinations valid and others not is more difficult to articulate in music. 
For example, the French word \emph{mode} can have its letters reorganized
into \emph{d\^{o}me} or \emph{d\'{e}mo} whereas \emph{doem} or \emph{medo}
are not words in the French lexicon.  Musical notes, on the other hand, can
be combined in any order and carry a subjective argument in musical time.
Several classic mathematic tools can be used to articulate musics
combinatorial potential: magic squares, Pascal's triangle, the Dragon
Curve (see below) and many others.
\section*{Attractors}
Attractors in music are a tool used by composers looking for a music that
contains qualities of both structure and chaos.

One interesting example of attractors in music comes from the Moscow-based
composer Sergei Zagny (born in 1960) and is based on mathematical equations
elaborated by the Belgian mathemetician Pierre-Fran\c{c}ois Verhulst
(1804-1849). A variable $x$ (which must always stay between 0 and 1) is
recursively multiplied by a factor $a$ (that can be between 0 and 4) and we
observe the following iterative results.

\begin{figure}[h]
\begin{lstlisting}
a = 2.6; x = 0.5; ct = 0;
While[ct <= 30, ct++; x = N[a*x*(1 - x)];
Print[ct, "  ", x]]

0.652 0.59153 0.6282324 0.6072475 0.6200956
0.6125017 0.6170938 0.6143529 0.61600210 
0.61501311 0.61560712 0.61525113 0.61546514
0.61533715 0.61541316 0.61536717 0.61539518 
0.61537819 0.61538820 0.61538221 0.61538622
0.61538423 0.61538524 0.61538425 0.61538526
0.61538527 0.61538528 0.61538529 0.61538530
0.61538531 0.615385
\end{lstlisting}
\caption{with $a=2.6$ and $x=0.5$ as a starting point, $x$ becomes 0.65 then 0.59. 0.62. 0.61
etc.
until it stabalizes at 0.615385.}
\end{figure}
\begin{figure}[h]
a = 3.56994; x = 0.5; ct = 0;
While[ct <= 30, ct++; x = N[a*x*(1 - x)];
Print[ct, "  ", x]]

0.8924852 0.3425553 0.8039914 0.5625865
0.8785026 0.3810437 0.8419688 0.4750099
0.89025510 0.34878611 0.81085612 0.54751813
0.88442414 0.36491215 0.82733816 0.50996617 
0.8921318 0.34354919 0.80510320 0.56016621
0.87956222 0.37817323 0.83950124 0.48101125 
0.89119826 0.34615727 0.80799328 0.55384229 
0.88213630 0.37117431 0.833238
\caption{a=3.56994, which is one of Feigenbaum's constants.  The result is
much more chaotic. In spite of the large differences between later
smaller significant figures, the system alternates between two poles -- one
close to 0.8 and the other close to 0.5.  The attractor thus becomes
periodic.}
\end{figure}
\begin{figure}[h]
En d’autres termes, les résultats tendent à s’organiser selon un cycle de quatre valeurs approximatives : tel est l’« attracteur » qui nous intéresse ici.
a = 3.745; x = 0.5; ct = 0;
While[ct <= 100, ct++; x = N[a*x*(1 - x)];
Print[ct, "  ", x]]

0.936252  0.2235243  0.6499864  0.8520045
0.472226  0.933367  0.2329368  0.6691459
0.82910610  0.53062711  0.93273712  0.23495613
0.6731714  0.82394615  0.54324616  0.92924617
0.24622518  0.69506619  0.7937520  0.61309721
0.88834822  0.37145123  0.87436424  0.41139425
0.90684826  0.31635927  0.80995328  0.57646429
0.91435430  0.29327331  0.77620432  0.65054933
0.8513734  0.47388935  0.93369736  0.23184237
0.66695238  0.83186539  0.52379640  0.93412941
0.23043642  0.66412143  0.83537644  0.51502445
0.93540546  0.22628347  0.65567148  0.84549549
0.4892250  0.93581551  0.22494552  0.65292153
0.84867454  0.48095855  0.93489256  0.22795457
0.65908658  0.8414759  0.49957660  0.93624961
0.22352662  0.6499963  0.85199864  0.47223465
0.93336366  0.23292767  0.66912668  0.82912969
0.5305770  0.9327571  0.23491372  0.67308573
0.82405574  0.54298175  0.92933276  0.2459577
0.69454378  0.79451379  0.61141780  0.8897681
0.36733582  0.87033883  0.42262184  0.91382785
0.29490886  0.77872687  0.64530888  0.85717689
0.45848390  0.92979591  0.2444692  0.69169993
0.79862694  0.6022895  0.89707396  0.34578797
0.84718898  0.4848399  0.935388100  0.226337101
0.655782
\caption{This configuration is less table and the numbers obtained are
completely dispersed.  Nevertheless, if one listens to the result in time,
one can hear an emergent logic in a random alteration between five disntinct
values.  In this way, music reveals what a base ten numerical system,
ill-adapted to the representation of dynamic system, often obfuscates.}
\end{figure}
\section*{Fun in \Tonnetz-land}
The \Tonnetz (tone-network) is a geometrical structure origianlly
introduced by the mathematician Leonhard Euler (1707-1783). In a treaty
entitled \emph{Speculum Musicum} (Musical Mirror), Euler propses a
representation of musical pitch in a two-dimensional space.  The space's two
axes are the perfect fifth (C to G, for example) and the major third (C to
E, for example).  After the octave, these are the two most ``consonant''
intervals, meaning that their ratios can be expressed with low integer
ratios.

The \Tonnetz utilized in contemporary musical discourse divides the
two-dimensional plane into traingles that correspond to major and minor
triadic chords.  Moving outward from the barycenter of the triangle
(corresponding, for example, to C-E-G), one notices three major axes of
symmetry that allow for paths through the \Tonnetz that are achieved
by changing one chordal note by one diatonic degree.
\begin{itemize}
\item The \emph{relative} chord (\emph{R}) -- C-E-A in the
above example.
\item The \emph{parallel} chord (\emph{P}) --
C-E$\flat$-G.
\item The \emph{leading-tone} chord (\emph{L}) --
B-E-G.
\end{itemize}
An interesting consequence of this partitioning is that major chords only
ever lead to minor chords and, inversely, minor chords only ever lead to
major chords.

From this triangulation of the musical plane, one obtains a hexagonal
lattice (a bee's nest as it were) where every note finds itself at the
center of a hexagon surrounded by six sommets (the six notes with which it
can form a major chord).  By the means of the three operators \emph{R},
\emph{P}, and \emph{L}, all of which preserve two out of three notes in a
three-note chord and change the moving note by no more than a whole tone,
one can navigate through the lattice to create harmonic progressions that
only require minimal voice-leading.

Of the many different paths that can be traced through the \Tonnetz,
some have particularly interesting qualities.  For example, from any given
chordal point on a hexagon in the \Tonnetz, it is possible to create a
path that exhausts all twenty-four major and minor chords with no
repetition.  We call this path a \emph{\Hamiltonian} traversal -- the word
\Hamiltonian coming from an analogoues operation in graph theory.  If a path
is cyclic, one calls it a ``\Hamiltonian cycle.''  An exhaustive research
shows that there are 124 \Hamiltonian cycles through the \Tonnetz that can be
classified according to their internal symmetries.  There are, for examples,
cycles that zig-zag because of two elemental symmetries within them (for
example, alternating \emph{L} and \emph{R} operations).  Other cycles
traverse the 24 major and minor chords without any internal symmetries.  The
figure below inventories the 28 \Hamiltonian \Tonnetz cycles that have
internal periodicity.  We find here, in positions 16 and 28, two zig-zag
configurations that allows one to visit each chords once and once only.

It is worth noting that all other symmetric operations, such as \emph{PR} or
\emph{LP}, create chord progressions that are shorter than 24.  These
non-\Hamiltonian cycles are shown below.

The classification of chord progressions in the \Tonnetz opens up a wide
array of possibilites in the study of musical harmonic organization.  With
respect to the OuMuPo, it gives way to new ways of writing music, especially
in the singer-songwriter sub-genre.  \emph{\Hamiltonian Songs} are now the
subject of concerts and workshops, and in 2016, a \Hamiltonian Cabaret,
emceed by Fabrice Guedy, Moreno Andreatta and the singer Polo, allowed
students of PSL (Paris Sciences et Lettres) to dive into this universe for
an entire week.  The students created a work entiteld \emph{Ballade-Marabout}
whose harmonic sructure is based on one of the periodic Hamiltonian cycles
discussed above.

\section*{A simple musical experiment: The Dragon}
Described by Martin Gardner in a 1967 generalist journal, the \textbf{Dragon
Curve} is an interesting musical object that is as simple as it is
fascinating.  While it has become a familiar and popular entry into
mathematical concepts, few have applied this curve to the creation of a
musical score.

We all know the small game of folding a slip of paper $n$ times in the same
direction and then unfolding the paper to see the final form.  The
complexity of the exercise increases according to the number of folds.  This
Dragon Curve possesses several interesting properties (fractal,
self-similar, symmetric but non-periodic, etc.).  The curve can also be read
as a series of melodic movements
\begin{enumerate}
\item Choose a scale, with a preference for non-standard scales such as C
D$\sharp$ E F G A$\flat$ B.
\item Fold a slip of paper twice in the same direction, then unfold it and
lie it down flat.
\item Play the first note of the scale, and for each ``bump'' in the paper,
go up one note.  For each dip in the paper, go down by one note (always
following the chosen scale).
\item Take the slip of paper, add an extra fold, and play the new melody
obtained by this change.
\end{enumerate}
\end{document}